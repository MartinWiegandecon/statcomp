% Options for packages loaded elsewhere
\PassOptionsToPackage{unicode}{hyperref}
\PassOptionsToPackage{hyphens}{url}
\PassOptionsToPackage{dvipsnames,svgnames,x11names}{xcolor}
%
\documentclass[
]{report}

\usepackage{amsmath,amssymb}
\usepackage[]{libertinus}
\usepackage{iftex}
\ifPDFTeX
  \usepackage[T1]{fontenc}
  \usepackage[utf8]{inputenc}
  \usepackage{textcomp} % provide euro and other symbols
\else % if luatex or xetex
  \usepackage{unicode-math}
  \defaultfontfeatures{Scale=MatchLowercase}
  \defaultfontfeatures[\rmfamily]{Ligatures=TeX,Scale=1}
\fi
% Use upquote if available, for straight quotes in verbatim environments
\IfFileExists{upquote.sty}{\usepackage{upquote}}{}
\IfFileExists{microtype.sty}{% use microtype if available
  \usepackage[]{microtype}
  \UseMicrotypeSet[protrusion]{basicmath} % disable protrusion for tt fonts
}{}
\makeatletter
\@ifundefined{KOMAClassName}{% if non-KOMA class
  \IfFileExists{parskip.sty}{%
    \usepackage{parskip}
  }{% else
    \setlength{\parindent}{0pt}
    \setlength{\parskip}{6pt plus 2pt minus 1pt}}
}{% if KOMA class
  \KOMAoptions{parskip=half}}
\makeatother
\usepackage{xcolor}
\usepackage[top = 30mm,left = 20mm]{geometry}
\setlength{\emergencystretch}{3em} % prevent overfull lines
\setcounter{secnumdepth}{-\maxdimen} % remove section numbering
% Make \paragraph and \subparagraph free-standing
\ifx\paragraph\undefined\else
  \let\oldparagraph\paragraph
  \renewcommand{\paragraph}[1]{\oldparagraph{#1}\mbox{}}
\fi
\ifx\subparagraph\undefined\else
  \let\oldsubparagraph\subparagraph
  \renewcommand{\subparagraph}[1]{\oldsubparagraph{#1}\mbox{}}
\fi

\usepackage{color}
\usepackage{fancyvrb}
\newcommand{\VerbBar}{|}
\newcommand{\VERB}{\Verb[commandchars=\\\{\}]}
\DefineVerbatimEnvironment{Highlighting}{Verbatim}{commandchars=\\\{\}}
% Add ',fontsize=\small' for more characters per line
\usepackage{framed}
\definecolor{shadecolor}{RGB}{241,243,245}
\newenvironment{Shaded}{\begin{snugshade}}{\end{snugshade}}
\newcommand{\AlertTok}[1]{\textcolor[rgb]{0.68,0.00,0.00}{#1}}
\newcommand{\AnnotationTok}[1]{\textcolor[rgb]{0.37,0.37,0.37}{#1}}
\newcommand{\AttributeTok}[1]{\textcolor[rgb]{0.40,0.45,0.13}{#1}}
\newcommand{\BaseNTok}[1]{\textcolor[rgb]{0.68,0.00,0.00}{#1}}
\newcommand{\BuiltInTok}[1]{\textcolor[rgb]{0.00,0.23,0.31}{#1}}
\newcommand{\CharTok}[1]{\textcolor[rgb]{0.13,0.47,0.30}{#1}}
\newcommand{\CommentTok}[1]{\textcolor[rgb]{0.37,0.37,0.37}{#1}}
\newcommand{\CommentVarTok}[1]{\textcolor[rgb]{0.37,0.37,0.37}{\textit{#1}}}
\newcommand{\ConstantTok}[1]{\textcolor[rgb]{0.56,0.35,0.01}{#1}}
\newcommand{\ControlFlowTok}[1]{\textcolor[rgb]{0.00,0.23,0.31}{#1}}
\newcommand{\DataTypeTok}[1]{\textcolor[rgb]{0.68,0.00,0.00}{#1}}
\newcommand{\DecValTok}[1]{\textcolor[rgb]{0.68,0.00,0.00}{#1}}
\newcommand{\DocumentationTok}[1]{\textcolor[rgb]{0.37,0.37,0.37}{\textit{#1}}}
\newcommand{\ErrorTok}[1]{\textcolor[rgb]{0.68,0.00,0.00}{#1}}
\newcommand{\ExtensionTok}[1]{\textcolor[rgb]{0.00,0.23,0.31}{#1}}
\newcommand{\FloatTok}[1]{\textcolor[rgb]{0.68,0.00,0.00}{#1}}
\newcommand{\FunctionTok}[1]{\textcolor[rgb]{0.28,0.35,0.67}{#1}}
\newcommand{\ImportTok}[1]{\textcolor[rgb]{0.00,0.46,0.62}{#1}}
\newcommand{\InformationTok}[1]{\textcolor[rgb]{0.37,0.37,0.37}{#1}}
\newcommand{\KeywordTok}[1]{\textcolor[rgb]{0.00,0.23,0.31}{#1}}
\newcommand{\NormalTok}[1]{\textcolor[rgb]{0.00,0.23,0.31}{#1}}
\newcommand{\OperatorTok}[1]{\textcolor[rgb]{0.37,0.37,0.37}{#1}}
\newcommand{\OtherTok}[1]{\textcolor[rgb]{0.00,0.23,0.31}{#1}}
\newcommand{\PreprocessorTok}[1]{\textcolor[rgb]{0.68,0.00,0.00}{#1}}
\newcommand{\RegionMarkerTok}[1]{\textcolor[rgb]{0.00,0.23,0.31}{#1}}
\newcommand{\SpecialCharTok}[1]{\textcolor[rgb]{0.37,0.37,0.37}{#1}}
\newcommand{\SpecialStringTok}[1]{\textcolor[rgb]{0.13,0.47,0.30}{#1}}
\newcommand{\StringTok}[1]{\textcolor[rgb]{0.13,0.47,0.30}{#1}}
\newcommand{\VariableTok}[1]{\textcolor[rgb]{0.07,0.07,0.07}{#1}}
\newcommand{\VerbatimStringTok}[1]{\textcolor[rgb]{0.13,0.47,0.30}{#1}}
\newcommand{\WarningTok}[1]{\textcolor[rgb]{0.37,0.37,0.37}{\textit{#1}}}

\providecommand{\tightlist}{%
  \setlength{\itemsep}{0pt}\setlength{\parskip}{0pt}}\usepackage{longtable,booktabs,array}
\usepackage{calc} % for calculating minipage widths
% Correct order of tables after \paragraph or \subparagraph
\usepackage{etoolbox}
\makeatletter
\patchcmd\longtable{\par}{\if@noskipsec\mbox{}\fi\par}{}{}
\makeatother
% Allow footnotes in longtable head/foot
\IfFileExists{footnotehyper.sty}{\usepackage{footnotehyper}}{\usepackage{footnote}}
\makesavenoteenv{longtable}
\usepackage{graphicx}
\makeatletter
\def\maxwidth{\ifdim\Gin@nat@width>\linewidth\linewidth\else\Gin@nat@width\fi}
\def\maxheight{\ifdim\Gin@nat@height>\textheight\textheight\else\Gin@nat@height\fi}
\makeatother
% Scale images if necessary, so that they will not overflow the page
% margins by default, and it is still possible to overwrite the defaults
% using explicit options in \includegraphics[width, height, ...]{}
\setkeys{Gin}{width=\maxwidth,height=\maxheight,keepaspectratio}
% Set default figure placement to htbp
\makeatletter
\def\fps@figure{htbp}
\makeatother
\newlength{\cslhangindent}
\setlength{\cslhangindent}{1.5em}
\newlength{\csllabelwidth}
\setlength{\csllabelwidth}{3em}
\newlength{\cslentryspacingunit} % times entry-spacing
\setlength{\cslentryspacingunit}{\parskip}
\newenvironment{CSLReferences}[2] % #1 hanging-ident, #2 entry spacing
 {% don't indent paragraphs
  \setlength{\parindent}{0pt}
  % turn on hanging indent if param 1 is 1
  \ifodd #1
  \let\oldpar\par
  \def\par{\hangindent=\cslhangindent\oldpar}
  \fi
  % set entry spacing
  \setlength{\parskip}{#2\cslentryspacingunit}
 }%
 {}
\usepackage{calc}
\newcommand{\CSLBlock}[1]{#1\hfill\break}
\newcommand{\CSLLeftMargin}[1]{\parbox[t]{\csllabelwidth}{#1}}
\newcommand{\CSLRightInline}[1]{\parbox[t]{\linewidth - \csllabelwidth}{#1}\break}
\newcommand{\CSLIndent}[1]{\hspace{\cslhangindent}#1}

\makeatletter
\makeatother
\makeatletter
\makeatother
\makeatletter
\@ifpackageloaded{caption}{}{\usepackage{caption}}
\AtBeginDocument{%
\ifdefined\contentsname
  \renewcommand*\contentsname{Table of contents}
\else
  \newcommand\contentsname{Table of contents}
\fi
\ifdefined\listfigurename
  \renewcommand*\listfigurename{List of Figures}
\else
  \newcommand\listfigurename{List of Figures}
\fi
\ifdefined\listtablename
  \renewcommand*\listtablename{List of Tables}
\else
  \newcommand\listtablename{List of Tables}
\fi
\ifdefined\figurename
  \renewcommand*\figurename{Figure}
\else
  \newcommand\figurename{Figure}
\fi
\ifdefined\tablename
  \renewcommand*\tablename{Table}
\else
  \newcommand\tablename{Table}
\fi
}
\@ifpackageloaded{float}{}{\usepackage{float}}
\floatstyle{ruled}
\@ifundefined{c@chapter}{\newfloat{codelisting}{h}{lop}}{\newfloat{codelisting}{h}{lop}[chapter]}
\floatname{codelisting}{Listing}
\newcommand*\listoflistings{\listof{codelisting}{List of Listings}}
\makeatother
\makeatletter
\@ifpackageloaded{caption}{}{\usepackage{caption}}
\@ifpackageloaded{subcaption}{}{\usepackage{subcaption}}
\makeatother
\makeatletter
\@ifpackageloaded{tcolorbox}{}{\usepackage[many]{tcolorbox}}
\makeatother
\makeatletter
\@ifundefined{shadecolor}{\definecolor{shadecolor}{rgb}{.97, .97, .97}}
\makeatother
\makeatletter
\makeatother
\ifLuaTeX
  \usepackage{selnolig}  % disable illegal ligatures
\fi
\IfFileExists{bookmark.sty}{\usepackage{bookmark}}{\usepackage{hyperref}}
\IfFileExists{xurl.sty}{\usepackage{xurl}}{} % add URL line breaks if available
\urlstyle{same} % disable monospaced font for URLs
\hypersetup{
  colorlinks=true,
  linkcolor={blue},
  filecolor={Maroon},
  citecolor={Blue},
  urlcolor={Blue},
  pdfcreator={LaTeX via pandoc}}

\author{}
\date{February 2, 2023}

\begin{document}
\ifdefined\Shaded\renewenvironment{Shaded}{\begin{tcolorbox}[enhanced, breakable, borderline west={3pt}{0pt}{shadecolor}, frame hidden, interior hidden, boxrule=0pt, sharp corners]}{\end{tcolorbox}}\fi

\begin{center}\vspace{0.3cm}
\textbf{\Large Modern Statistical Computing} \\
\vspace{5pt} {\large Seminar \#2} \\
\vspace{5pt} {\large February 2, 2023} \\
\vspace{5pt} {\large Homework part due on February 9, 2023 (3pm)} 
\end{center}

\flushleft

\rule{\linewidth}{0.1mm}

In this seminar, you will be studying the long-term effects of the
Spanish Inquisition by replicating some of the results in Drelichman,
Vidal-Robert, and Voth (2021). The Spanish Inquisition existed from 1478
to 1843 and was responsible for combatting heresy (= deviation from
Catholic doctrine), prosecuting tens of thousands of
individuals.\footnote{A more detailed historical account can be found on
  page 2 in Drelichman, Vidal-Robert, and Voth (2021).}

Before starting to work on the data, think about possible ways how a
historic institution like the Spanish Inquisition can affect economic
and social outcomes today.

\subsection{Exploring the Data}

\begin{enumerate}
\def\labelenumi{\arabic{enumi}.}
\tightlist
\item
  Load the data into R. The key variables of interest are:
\end{enumerate}

\begin{itemize}
\tightlist
\item
  \texttt{adj\_impact}: Measure for the strength of the inquisition
  \footnote{More precisely, it is defined as the number of years when
    the Inquisition persecuted at least one community member as a share
    of all years observed.}
\end{itemize}

\begin{itemize}
\item
  \texttt{gdppc}: GDP p.c.
\item
  \texttt{religious}: Number of survey respondents who attended
  religious services in the previous week
\item
  \texttt{c\_secondplus}: Population share with a high school degree or
  higher
\item
  \texttt{trust2}: A (standardized) measure of trust from surveys

  The dataset also includes further sociodemographic and geographic
  variables which we will solely use as control variables in our
  regressions later.\\
  Make a table including summary statistics on the key variables listed
  above (you can for instance use the functions \texttt{describe} or
  \texttt{stargazer} from the packages \texttt{HMisc} and
  \texttt{stargazer} package, respectively).
\end{itemize}

\begin{enumerate}
\def\labelenumi{\arabic{enumi}.}
\setcounter{enumi}{1}
\tightlist
\item
  In which autonomous regions was the Spanish Inquisition the strongest?
  To get a sense of the geographic distribution of the impact of the
  Spanish Inquisition, generate a map showing the geographic
  distribution of its strength (you can either plot \texttt{adj\_impact}
  directly, or first generate a factor-variable that contains quintile
  of \texttt{adj\_impact}). You can find the corresponding shapefile
  \texttt{municoordinates.csv} on AulaGlobal.
\end{enumerate}

To help you generate this plot, you can find some example code below:

\begin{Shaded}
\begin{Highlighting}[]
\CommentTok{\# Color palette used }
\FunctionTok{require}\NormalTok{(RColorBrewer)}
\NormalTok{blues }\OtherTok{\textless{}{-}} \FunctionTok{brewer.pal}\NormalTok{(}\DecValTok{9}\NormalTok{, }\StringTok{"Blues"}\NormalTok{)}
\NormalTok{blue\_palette }\OtherTok{\textless{}{-}} \FunctionTok{c}\NormalTok{(blues[}\DecValTok{2}\NormalTok{], blues[}\DecValTok{4}\NormalTok{], blues[}\DecValTok{6}\NormalTok{], blues[}\DecValTok{8}\NormalTok{], blues[}\DecValTok{9}\NormalTok{])}

\FunctionTok{ggplot}\NormalTok{() }\SpecialCharTok{+} 
  \FunctionTok{geom\_polygon}\NormalTok{(}\AttributeTok{data =}\NormalTok{ municoord\_df, }\FunctionTok{aes}\NormalTok{(}\AttributeTok{fill =}\NormalTok{ quintiles, }
               \AttributeTok{x =}\NormalTok{ x, }\AttributeTok{y =}\NormalTok{ y, }\AttributeTok{group =}\NormalTok{ munid), }
               \AttributeTok{size =} \DecValTok{0}\NormalTok{, }\AttributeTok{color =} \StringTok{"gray45"}\NormalTok{, }\AttributeTok{alpha =} \DecValTok{1}\NormalTok{) }\SpecialCharTok{+} 
  \FunctionTok{scale\_fill\_manual}\NormalTok{(}\AttributeTok{values =}\NormalTok{ blue\_palette, }\AttributeTok{name =} \StringTok{"Inquisition Impact }\SpecialCharTok{\textbackslash{}n}\StringTok{ Quintile"}\NormalTok{, }
                    \AttributeTok{breaks =} \FunctionTok{c}\NormalTok{(}\DecValTok{1}\NormalTok{, }\DecValTok{2}\NormalTok{, }\DecValTok{3}\NormalTok{, }\DecValTok{4}\NormalTok{, }\DecValTok{5}\NormalTok{)) }\SpecialCharTok{+}
  \FunctionTok{theme\_void}\NormalTok{() }\SpecialCharTok{+} 
  \FunctionTok{coord\_quickmap}\NormalTok{()}
\end{Highlighting}
\end{Shaded}

\subsection{Assessing the Long-Run Effects of the Spanish Inquisition}

We are mainly interested in the effects of the Spanish Inquisition on
today's economic development. Even though we won't be able to pin down a
causal link from the Spanish Inquisition to economic development, we can
at least assess their correlation and rule out some alternative
explanations.

\begin{enumerate}
\def\labelenumi{\arabic{enumi}.}
\item
  Generate a variable that assigns each municipality to one of the
  following categories:

  \begin{itemize}
  \tightlist
  \item
    \texttt{adj\_impact} missing or 0
  \item
    bottom tercile of non-zero \texttt{adj\_impact} values
  \item
    middle tercile of non-zero \texttt{adj\_impact} values
  \item
    top tercile of non-zero \texttt{adj\_impact} values
  \end{itemize}

  Show the distribution of GDP p.c. by group. Do municipalities with
  larger impact of the Spanish Inquisition have lower or higher GDP per
  capita than less impacted municipalities?
\item
  Before assessing the relationship between GDP p.c. and the Spanish
  Inquisition, check if the dependent variable (GDP p.c.) follows
  (roughly) a normal distribution? How would you transform the variable
  to use it in the regression later? Throughout the rest of the
  assignment, only use the transformed GDP p.c. variable. Next, regress
  your transformed GDP p.c. variable on \texttt{adj\_impact}. How do you
  interpret the coefficient? Plot the residuals against
  \texttt{adj\_impact}. Do you have to adjust the standard errors?
\item
  We would like to examine if the relationship between GDP p.c. and
  \texttt{adj\_impact}, documented in 2., might be driven by other
  sociodemographic or geographic confounders. To do so, add
  (step-by-step) the following variables to the linear model estimated
  in 2.

  \begin{itemize}
  \tightlist
  \item
    Sociodemographic controls: Log population, average age, the upper
    class population share and the share married
  \item
    Geographic controls: Latitude, longitude, distance to river,
    distance to river
  \item
    Sociodemographic + geographic controls
  \item
    Autonomous region fixed effects + sociodemographic and geogrpahic
    (you can use the \href{https://lrberge.github.io/fixest/}{fixest}
    package and its function \texttt{feols})
  \end{itemize}
\end{enumerate}

Show, graphically or in a table, whether the \texttt{adj\_impact}
coefficient changes across the different specifications.

\subsection{Studying other Outcomes (Homework, due: 09-02-2023, 3pm)}

\emph{Remark:} Please submit your homework, ideally as a PDF document,
via email to
\href{mailto:martin.wiegand@upf.edu}{\nolinkurl{martin.wiegand@upf.edu}},

\begin{enumerate}
\def\labelenumi{\arabic{enumi}.}
\item
  Write a loop that estimates the full linear model (with all controls
  and autonomous region FEs) for the following outcomes: religious,
  education and trust. Generate a coefficient plot showing the effect of
  \texttt{adj\_impact} on each of the outcomes (incl.~GDP p.c. from part
  2).
\item
  Finally, we are interested whether the effect of the Spanish
  Inquisition on economic development varies by region. To assess this
  hypothesis, estimate the model including control variables and let the
  \texttt{adj\_impact} effect vary by autonomous region. Is the effect
  of the Spanish Inquisition on today's economic development homogeneous
  across regions?
\end{enumerate}

\pagebreak

\subsection{References}

\hypertarget{refs}{}
\begin{CSLReferences}{1}{0}
\leavevmode\vadjust pre{\hypertarget{ref-Drelichman_etal_2021_Inquisition}{}}%
Drelichman, Mauricio, Jordi Vidal-Robert, and Hans-Joachim Voth. 2021.
{``The Long-Run Effects of Religious Persecution: Evidence from the
Spanish Inquisition.''} \emph{Proceedings of the National Academy of
Sciences} 118 (33): e2022881118.
\url{https://doi.org/10.1073/pnas.2022881118}.

\end{CSLReferences}



\end{document}
