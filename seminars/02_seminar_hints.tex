% Options for packages loaded elsewhere
\PassOptionsToPackage{unicode}{hyperref}
\PassOptionsToPackage{hyphens}{url}
\PassOptionsToPackage{dvipsnames,svgnames,x11names}{xcolor}
%
\documentclass[
]{report}

\usepackage{amsmath,amssymb}
\usepackage[]{libertinus}
\usepackage{iftex}
\ifPDFTeX
  \usepackage[T1]{fontenc}
  \usepackage[utf8]{inputenc}
  \usepackage{textcomp} % provide euro and other symbols
\else % if luatex or xetex
  \usepackage{unicode-math}
  \defaultfontfeatures{Scale=MatchLowercase}
  \defaultfontfeatures[\rmfamily]{Ligatures=TeX,Scale=1}
\fi
% Use upquote if available, for straight quotes in verbatim environments
\IfFileExists{upquote.sty}{\usepackage{upquote}}{}
\IfFileExists{microtype.sty}{% use microtype if available
  \usepackage[]{microtype}
  \UseMicrotypeSet[protrusion]{basicmath} % disable protrusion for tt fonts
}{}
\makeatletter
\@ifundefined{KOMAClassName}{% if non-KOMA class
  \IfFileExists{parskip.sty}{%
    \usepackage{parskip}
  }{% else
    \setlength{\parindent}{0pt}
    \setlength{\parskip}{6pt plus 2pt minus 1pt}}
}{% if KOMA class
  \KOMAoptions{parskip=half}}
\makeatother
\usepackage{xcolor}
\usepackage[top = 30mm,left = 20mm]{geometry}
\setlength{\emergencystretch}{3em} % prevent overfull lines
\setcounter{secnumdepth}{-\maxdimen} % remove section numbering
% Make \paragraph and \subparagraph free-standing
\ifx\paragraph\undefined\else
  \let\oldparagraph\paragraph
  \renewcommand{\paragraph}[1]{\oldparagraph{#1}\mbox{}}
\fi
\ifx\subparagraph\undefined\else
  \let\oldsubparagraph\subparagraph
  \renewcommand{\subparagraph}[1]{\oldsubparagraph{#1}\mbox{}}
\fi


\providecommand{\tightlist}{%
  \setlength{\itemsep}{0pt}\setlength{\parskip}{0pt}}\usepackage{longtable,booktabs,array}
\usepackage{calc} % for calculating minipage widths
% Correct order of tables after \paragraph or \subparagraph
\usepackage{etoolbox}
\makeatletter
\patchcmd\longtable{\par}{\if@noskipsec\mbox{}\fi\par}{}{}
\makeatother
% Allow footnotes in longtable head/foot
\IfFileExists{footnotehyper.sty}{\usepackage{footnotehyper}}{\usepackage{footnote}}
\makesavenoteenv{longtable}
\usepackage{graphicx}
\makeatletter
\def\maxwidth{\ifdim\Gin@nat@width>\linewidth\linewidth\else\Gin@nat@width\fi}
\def\maxheight{\ifdim\Gin@nat@height>\textheight\textheight\else\Gin@nat@height\fi}
\makeatother
% Scale images if necessary, so that they will not overflow the page
% margins by default, and it is still possible to overwrite the defaults
% using explicit options in \includegraphics[width, height, ...]{}
\setkeys{Gin}{width=\maxwidth,height=\maxheight,keepaspectratio}
% Set default figure placement to htbp
\makeatletter
\def\fps@figure{htbp}
\makeatother
\newlength{\cslhangindent}
\setlength{\cslhangindent}{1.5em}
\newlength{\csllabelwidth}
\setlength{\csllabelwidth}{3em}
\newlength{\cslentryspacingunit} % times entry-spacing
\setlength{\cslentryspacingunit}{\parskip}
\newenvironment{CSLReferences}[2] % #1 hanging-ident, #2 entry spacing
 {% don't indent paragraphs
  \setlength{\parindent}{0pt}
  % turn on hanging indent if param 1 is 1
  \ifodd #1
  \let\oldpar\par
  \def\par{\hangindent=\cslhangindent\oldpar}
  \fi
  % set entry spacing
  \setlength{\parskip}{#2\cslentryspacingunit}
 }%
 {}
\usepackage{calc}
\newcommand{\CSLBlock}[1]{#1\hfill\break}
\newcommand{\CSLLeftMargin}[1]{\parbox[t]{\csllabelwidth}{#1}}
\newcommand{\CSLRightInline}[1]{\parbox[t]{\linewidth - \csllabelwidth}{#1}\break}
\newcommand{\CSLIndent}[1]{\hspace{\cslhangindent}#1}

\makeatletter
\makeatother
\makeatletter
\makeatother
\makeatletter
\@ifpackageloaded{caption}{}{\usepackage{caption}}
\AtBeginDocument{%
\ifdefined\contentsname
  \renewcommand*\contentsname{Table of contents}
\else
  \newcommand\contentsname{Table of contents}
\fi
\ifdefined\listfigurename
  \renewcommand*\listfigurename{List of Figures}
\else
  \newcommand\listfigurename{List of Figures}
\fi
\ifdefined\listtablename
  \renewcommand*\listtablename{List of Tables}
\else
  \newcommand\listtablename{List of Tables}
\fi
\ifdefined\figurename
  \renewcommand*\figurename{Figure}
\else
  \newcommand\figurename{Figure}
\fi
\ifdefined\tablename
  \renewcommand*\tablename{Table}
\else
  \newcommand\tablename{Table}
\fi
}
\@ifpackageloaded{float}{}{\usepackage{float}}
\floatstyle{ruled}
\@ifundefined{c@chapter}{\newfloat{codelisting}{h}{lop}}{\newfloat{codelisting}{h}{lop}[chapter]}
\floatname{codelisting}{Listing}
\newcommand*\listoflistings{\listof{codelisting}{List of Listings}}
\makeatother
\makeatletter
\@ifpackageloaded{caption}{}{\usepackage{caption}}
\@ifpackageloaded{subcaption}{}{\usepackage{subcaption}}
\makeatother
\makeatletter
\@ifpackageloaded{tcolorbox}{}{\usepackage[many]{tcolorbox}}
\makeatother
\makeatletter
\@ifundefined{shadecolor}{\definecolor{shadecolor}{rgb}{.97, .97, .97}}
\makeatother
\makeatletter
\makeatother
\ifLuaTeX
  \usepackage{selnolig}  % disable illegal ligatures
\fi
\IfFileExists{bookmark.sty}{\usepackage{bookmark}}{\usepackage{hyperref}}
\IfFileExists{xurl.sty}{\usepackage{xurl}}{} % add URL line breaks if available
\urlstyle{same} % disable monospaced font for URLs
\hypersetup{
  colorlinks=true,
  linkcolor={blue},
  filecolor={Maroon},
  citecolor={Blue},
  urlcolor={Blue},
  pdfcreator={LaTeX via pandoc}}

\author{}
\date{February 5, 2023}

\begin{document}
\ifdefined\Shaded\renewenvironment{Shaded}{\begin{tcolorbox}[breakable, sharp corners, borderline west={3pt}{0pt}{shadecolor}, interior hidden, frame hidden, enhanced, boxrule=0pt]}{\end{tcolorbox}}\fi

\begin{center}\vspace{0.3cm}
\textbf{\Large Modern Statistical Computing} \\
\vspace{5pt} {\large Hints for Seminar \#2} \\
\vspace{5pt} {\large February 5, 2023} \\
\end{center}

\flushleft

\rule{\linewidth}{0.1mm}

\subsection{Exploring the Data}

Some of you struggled with correctly reading the data into R. In many
cases, the data types for variables were not correct,
e.g.~\texttt{adj\_impact} was read in as a character even though it is
numeric. From my experience, this happens more often when using
\texttt{read.csv} and not \texttt{read\_csv} (which is from the
\texttt{readr} package). I would recommend using \texttt{read\_csv}.
That way, the data type is more likely to be correct and you have the
dataset right away as a tibble, which makes it easier to inspect it.

As many of you have noticed, it is easier to use the
\texttt{describe}function from \texttt{HMisc}instead of
\texttt{stargazer}to generate the summary statistics. As a side remark,
the latter (\texttt{stargazer}) can be quite helpful if you want to
export tables from R into a .tex document, for instance.

\subsection{Assessing the Long-Run Effects of the Spanish Inquisition}

\begin{enumerate}
\def\labelenumi{\arabic{enumi}.}
\item
  Again, there are multiple ways to tackle this. One would be to first
  generate a only takes the positive values of \texttt{adj\_impact}(it
  will be NA for \texttt{adj\_impact}-values of zero), then its
  corresponding tercile variable, and ultimately generate a factor
  variable that is equal to 0 if \texttt{adj\_impact} equals 0 (or NA),
  and otherwise equals the newly generated tercile variable.
\item
  If you want to double check that you used the ``correct''
  transformation of the dependent variable, you can check your solution
  with Drelichman, Vidal-Robert, and Voth (2021). As written in the
  instructions, only use the transformed variable in the following
  exercises.
\item
  When I ask you to estimate different linear models
  \emph{step-by-step}, I mean that you proceed as follows:

  \begin{enumerate}
  \def\labelenumii{\arabic{enumii}.}
  \tightlist
  \item
    Estimate the linear model with no control variables, i.e.~estimate
    \[
     \text{Transformed GDP p.c.}_m = \beta_0 + \beta_1 \text{Adjusted Impact}_m + \epsilon_m 
     \]
  \item
    Add sociodemographic control variables to the model above: \[
     \text{Transformed GDP p.c.}_m = \beta_0 + \beta_1 \text{Adjusted Impact}_m + \text{Sociodemographic Controls}_m' \gamma +  \epsilon_m 
     \]
  \item
    Add geographic (instead of sociodemographic) controls to the
    baseline model from 1.
  \item
    Have both control variable groups in the model.
  \item
    Finally, add autonomous region fixed effects. I again suggest that
    you take a look at \href{https://lrberge.github.io/fixest/}{fixest}
    package and its function \texttt{feols}.
  \end{enumerate}
\end{enumerate}

\subsection{Studying other Outcomes (Homework, due: 09-02-2023, 3pm)}

\begin{enumerate}
\def\labelenumi{\arabic{enumi}.}
\item
  Estimate the model from bullet point 5. above for other outcomes,
  i.e.~change the left-hand-side of the regression equation
  (e.g.~instead of using Transformed GDP p.c.\(_m\), use
  e.g.~Religious\(_m\)).
\item
  Here, you are supposed to estimate the model from bullet point 4.,
  \emph{but} let the coefficient on \texttt{adj\_impact} vary by
  autonomous region. To be precise, you are supposed to have an
  interaction variable (or multiple, depending on how you code it)
  between a factor variable of the autonomous region and the
  \texttt{adj\_impact} variable, such that you end up with an estimate
  of the effect of \texttt{adj\_impact}on GDP p.c. for every autonomous
  regions.
\end{enumerate}

\pagebreak

\hypertarget{refs}{}
\begin{CSLReferences}{1}{0}
\leavevmode\vadjust pre{\hypertarget{ref-Drelichman_etal_2021_Inquisition}{}}%
Drelichman, Mauricio, Jordi Vidal-Robert, and Hans-Joachim Voth. 2021.
{``The Long-Run Effects of Religious Persecution: Evidence from the
Spanish Inquisition.''} \emph{Proceedings of the National Academy of
Sciences} 118 (33): e2022881118.
\url{https://doi.org/10.1073/pnas.2022881118}.

\end{CSLReferences}



\end{document}
